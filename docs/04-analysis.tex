\chapter{Аналитический раздел}

Веб-сервер --- это программное средство, обеспечивающее пользователям сети доступ к гипертекстовым документам, расположенным на конкретном веб-узле~\cite{yeager1996web}. Веб-сервер принимает HTTP~\cite{touch1998analysis} запросы от клиентов и отправляет им HTTP-ответы. Наиболее популярными веб-серверами являются Apache и Nginx~\cite{putro2022comparison}.

Когда запрос от клиента поступает на веб-сервер, сервер использует специальные модули и конфигурации для обработки запроса и передачи статических файлов, таких как HTML, CSS, JavaScript, изображения и другие ресурсы.

Обычно веб-сервер на Linux предоставляет статические файлы, используя свои встроенные функции и механизмы обработки запросов, а также взаимодействует с ядром операционной системы для чтения файлов с диска и передачи их по сети клиенту. С помощью специальных модулей и конфигураций веб-сервер настраивается для эффективного отдачи статики~\cite{putro2022comparison}.

\section{Cокеты BSD}

Для передачи данных между множеством потоков, запущенных на одном или нескольких компьютерах, связанных между собой сетью, используется стек протоколов TCP/IP~\cite{tian2023building}. Сокеты представляют собой абстракцию конечной точки сетевого взаимодействия, которая, в случае TCP/IP, объединяет в виде структуры IP-адрес и номер порта~\cite{stivens}.

Сокеты были впервые введены в UNIX BSD как универсальное средство взаимодействия процессов как на отдельно взятой машине, так и в распределенных системах. Сокет создается системным вызовом \texttt{socket()}. На рисунке~\ref{img:socket2} представлена упрощенная схема взаимодействия двух процессов с помощью сокетов.

\includeimage
    {socket2}
    {f}
    {h}
    {0.7\linewidth}
    {Схема взаимодействия процессов с помощью сокетов}

На рисунке~\ref{img:syscalls} представлена последовательность системных вызовов, используемых на стороне клиента и на стороне сервера для обмена данными между процессами.

\includeimage
    {syscalls}
    {f}
    {h}
    {0.9\linewidth}
    {Системные вызовы для взаимодействия процессов с помощью сокетов}

Адреса и номера портов сокетов BSD должны быть указаны в сетевом порядке байтов~\cite{stivens}. 

\clearpage

\section{Стек протоколов TCP/IP}

Обычно общий термин <<TCP/IP>> означает все, что связано с конкретными протоколами TCP и IP. Это может включать другие протоколы, приложения, а также среду сети. Примерами таких протоколов могут быть UDP, ARP и ICMP. Примерами таких приложений могут быть TELNET, FTP и rcp.

Протокол TCP предназначен для разбивки сообщения на дейтаграммы и соединения их в конечной точке маршрута. TCP обеспечивает надежную, упорядоченную и проверенную доставку потока восьмибитовых байтов между приложениями, работающими на хостах, сообщающихся с помощью IP-сети~\cite{hunt2002tcp}. 

Протокол IP (Internet Protocol - межсетевой протокол) является маршрутизируемым протоколом сетевого уровня стека протоколов TCP/IP. Этот протокол предназначен для объединения отдельных компьютерных сетей во всемирную сеть Интернет. Самой важной функцией этого протокола является адресация сети~\cite{hunt2002tcp}. Задачей протокола IP является доставка пакетов от исходного узла на предназначенный узел только с помощью функции IP-адресация в заголовках пакетов. Для этого, IP определяет структуры пакетов, которые инкапсулируют данные. Он также определяет методы адресации, которые используются, чтобы обозначать дейтаграмму с исходными и предназначенными
информациями. IP отвечает за адресации хостов и для маршрутизации пакета от исходного узла к предназначенному узлу через один или несколько IP-сетей.

\clearpage

На рисунке \ref{img:tcp} приведены состояния сеанса TCP.

\includeimage
    {tcp}
    {f}
    {h}
    {0.65\linewidth}
    {Состояния сеанса TCP}

\section{Протокол HTTP}

HTTP (англ. HyperText Transfer Protocol) --- протокол прикладного уровня модели OSI/ISO, предназначенный для передачи гипертекстовых документов \cite{gourley2002http}. Протокол HTTP предполагает использование клиент-серверной структуры передачи данных. Клиентское приложение формирует запрос и отправляет его на сервер, после чего серверное программное обеспечение обрабатывает данный запрос, формирует ответ и передаёт его обратно клиенту. После этого клиентское приложение может продолжить отправлять другие запросы, которые будут обработаны аналогичным образом.

\clearpage

Так как HTTP --- клиент-серверный протокол, то запросы отправляются какой-то одной стороной (участником обмена, либо прокси). Чаще всего в качестве участника обмена выступает веб-браузер.

Каждый запрос отправляется серверу, обрабатывающему его и возвращающему ответ. Между клиентом и сервером на пути посылки запросов и ответов может быть множество посредников: определенных сетевых устройств, например, маршрутизаторов. Все эти посредники оперируют на сетевом и транспортном уровнях.

\subsection{Методы HTTP}

Метод определяет операцию, которую нужно осуществить с указанным ресурсом. Спецификация HTTP 1.1 не ограничивает количество методов, однако используются лишь некоторые, наиболее стандартные методы:

\begin{itemize}[label=---]
    \item OPTIONS --- запрос методов сервера (Allow);
    \item GET --- запрос документа;
    \item HEAD --- аналог GET, но без тела ответа;
    \item POST --- передача данных от клиента;
    \item PUT --- размещение файла по URI/изменение данных;
    \item PATCH --- частичное изменение информации на веб-сервере;
    \item DELETE --- удаление файла по URI/удаление данных.
\end{itemize}

URI (англ. Uniform Resource Identifier) --- путь до конкретного ресурса, над которым необходимо осуществить операцию (метод HTTP).
